\documentclass[a4paper]{article}
\usepackage[slovene]{babel}
\usepackage[utf8]{inputenc}
\usepackage[T1]{fontenc}
\usepackage{graphicx}
\usepackage{marvosym}
\usepackage{amssymb,amsmath}
\title{Verjetnost sekanja konveksne podmnožice}
\author{Peter Dolenc, Jan Rems \\ Finančni praktikum \\ Finančna matematika, Fakulteta za matematiko in fiziko}
\date{2017}
\newtheorem{definicija}{Definicija}
\newtheorem{izrek}{Izrek}


\begin{document}
\title{%
  Verjetnost sekanja konveksne podmnožice -\\
   \large Kratko poročilo \\}

\author{Peter Dolenc, Jan Rems}

\maketitle

\pagebreak

\section{Uvod}


V nalogi se bova soočila z naslednjim problemom. Zamislimo si, da imamo na Evklidski ravnini neko \textit{konveksno množico} $C$ , ki vsebuje \textit{konveksno podmnožico } $C'$. Najina naloga bo določiti verjetnost dogodka, da naključna premica, ki seka množico $C$ seka tudi podmnožico $C'$. Gre za klasični problem s področja stohastične geometrije, ki se ga da posplošiti tudi na druge veje verjetnostne teorije. Hipoteza, ki jo postavljava trdi, \textit{da bo verjetnost zgoraj omenjenega dogodka enaka razmerju obsegov konveksnih množic}. Najina glavna naloga bo, da postavljeno hipotezo najprej \textit{teoretično} utemeljiva, nato pa izvedeva še \textit{eksperiment}, kjer se bova s pomočjo programa, ki bo generiral naključne premice in konveksne množice, poizkusila čim bolj približati željenemu rezultatu. 


\vspace{4 mm}


\section{Analitična rešitev problema}

Za začetek vpeljimo nekaj pojmov, ki so potrebni za formuliranje klasičnega izreka s področja integralske geometrije, ki nam bo v pomoč pri izpeljavi rezultata, ki ga predvideva najina hipoteza.

\begin{definicija}
Množico vseh neorientiranih premic na $\mathbb{R}^2$ definiramo kot: $$ \{ (p,\theta)|0 \le \theta \le 2 \pi , p \ge 0\},$$ kjer $p$ predstavlja oddaljenost premice od izhodišča, $\theta$ pa kot premice glede na $x$ os.
\end{definicija}

\vspace{3 mm}

S tem ko imamo definirano množico vseh premic na $\mathbb{R}^2$, lahko na tej množici podamo tudi \textit{mero} s  predpisom $ \int \int d\theta dp$. Sedaj lahko podamo željeni izrek.

\begin{izrek}[Cauchy-Croftonova formula]
 Naj bo $c$ regularna ravninska krivulja in $n_c (p, \theta)$  število točk, pri katerih premica, parametrizirana na način, kot je podan v definiciji, seka krivuljo $c$. Potem za dolžino krivulje $L(c)$ veja: $$L(c) = \frac{1}{2}\int \int n_c (p, \theta) d\theta dp $$
 \end{izrek}

\vspace{3 mm}

Zaradi zgornjega rezultata je iskanje preseka premice s konveksno množico smiselno prevesti na problem iskanja presečišč premice in sklenjene regularne krivulje, ki omejuje dano konveksno množico. Tako nas namesto razmerja obsegov zanima razmerje dolžine krivulj, kjer ena omejuje konveksno množico $C$, druga pa njeno konveksno podmnožico $C'$. Tudi krivulji bomo ustrezno poimenovali $\gamma$ in $\gamma '$. Krivuljam, ki omejujejo konveksne množice, pravimo konveksne krivulje. 

Glavna naloga v teoretičnem delu naloge bo torej apliciranje Cauchy-Croftonove formule na primer konveksnih krivulj, čemur bo sledila izpeljava izražave verjetnosti dogodka, da premica ki seka krivuljo $\gamma$, seka tudi krivuljo $\gamma'$, z razmerjem dolžine obravnavanih krivulj. 



\section{Načrt eksperimentalnega dela}
Eksperimentalni del projekta se bova lotila z metodami Monte Carlo. To so simulacijske metode, ki s pomočjo naključnih števil in velikega števila izračunov in ponavljanja omogočajo predvidevanje obnašanja zapletenih matemtičnih sistemov. Ideja je preprosta: generirati dve naključni konveksni krivulji in pa naključno premico, ki seka večjo od obeh ter ugotoviti, ali premica seka tudi manjšo. Če tak poskus ponovimo velikokrat, bomo lahko ocenili verjetnost, da naključna premica, ki seka večjo konveksno krivuljo, seka tudi manjšo.

Na začetku bova poskušala preveriti najino hipotezo na bolj enostavnih konveksnih krivuljah - elipsah, kasneje pa se bova spopadla tudi z bolj splošnimi konveksnimi krivuljami. Pri delu z elipsami je cilj sprogramirati program, ki bo sposoben obvladovanja osnovnih prijemov: generiranje krivulj in premic in pa prepoznavanja, kdaj te premice res sekajo naše objekte. Ta orodja nama bodo pomagala kasneje, ko bova iz enostavnih elips želela preiti na bolj splošne in zapletene konveksne krivulje.

Drugi del izziva bo predstavljal generiranje naključnih premic. Za to obstaja nekaj različnih enakovrednih načinov. Eden takšnih je tak, kot je opisano v definicji: naključno določimo par $(p,\theta)$, kjer $p$ predstavlja oddaljenost premice do izhodišča, $\theta$ pa kot med premico in $x$ osjo. Težava takega generiranja premic je, da premice morda ne bodo sekale niti večjega od naših likov. Zato bova poskusila tudi tako, da v večji od najinih množic naključno generirava eno točko, drugo pa generirava poljubno. Taki dve točki enolično določita premico, ki teče skozi njiju.

Po opravljenih simulacijah pričakujeva, da bova opazila lastnost, ki jo bova pokazala v teoretičnem delu naloge - namreč da je verjetnost, da premica seka tudi manjšo krivuljo, enaka razmerju obsegov konveksnih krivulj. Zanimalo naju bo tudi, kako hitro bova dobila relativno natančno oceno za to verjetnost - torej kolikokrat morava poskus ponoviti, da bo verjetnost dobljena s poskusi blizu teoretične. Končno se nama zdi pomembno tudi, da najini algoritmi delujejo učinkovito, saj lahko pri velikem številu ponovitev poskusa pride do velike časovne zahtevnosti pri izvedbi programa. 

Delo bo potekalo v programskem okolju Matlab.

 




\end{document}