\documentclass[a4paper]{article}
\usepackage[slovene]{babel}
\usepackage[utf8]{inputenc}
\usepackage[T1]{fontenc}
\usepackage{graphicx}
\usepackage{marvosym}
\usepackage{amssymb,amsmath}
\title{Verjetnost sekanja konveksne podmnožice}
\author{Peter Dolenc, Jan Rems \\ Finančni praktikum \\ Finančna matematika, Fakulteta za matematiko in fiziko}
\date{2017}
\newtheorem{definicija}{Definicija}[section]
\newtheorem{izrek}{Izrek}


\begin{document}
\title{%
  Upravljanje omejenega maloprodajnega prostora za osnovne izdelke \\
  \large Kratko poročilo \\}

\author{Peter Dolenc, Jan Rems}

\maketitle

\pagebreak

\section{Uvod}


V nalogi se bova soočila z naslednjim problemom. Zamislimo si, da imamo v ravnini neko \textit{konveksno množico} $C$ , ki vsebuje \textit{konveksno podmnožico } $C'$. Najina naloga bo določiti verjetnost dogodka, da naključna premica, ki seka množico $C$ seka tudi podmnožico $C'$. Gre za klasični problem s področja stohastične geometrije, ki se ga da posplošiti tudi na druge veje verjetnostne teorije. Iz klasičnega rezultata integralske geometrije pričakujemo, da bo verjetnost zgoraj omenjenega dogodka enaka razmerju obsegov konveksnih množic. Najina glavna naloga bo, izvedba eksperimenta, kjer se bova z generiranjem naključnih premic in konveksnih množic, poizkusila čim bolj približati željenemu rezultatu. To nameravava doseči na eni strani z razumevanjem teoretičnega ozadja problema, na drugi pa z izvedbu čim natančnejšega eksperimenta.

\vspace{4 mm}


\section{Analitični pristop k problemu}

Kot bomo videli, se izkaže, da je iskanje preseka premice s konveksno množico, smiselno prevesti na problem iskanja presečišč premice in sklenjene regularne krivulje, ki omejuje dano konveksno množico. Tako nas namesto razmerja obsegov, zanima razmerje dolžine krivulj, kjer ena omejuje konveksno množico $C$, druga pa njeno konveksno podmnožico $C'$. Tudi krivulji bomo ustrezno poimenovali $\gamma$ in $\gamma '$. Krivuljam, ki omejujejo konveksne množice pravimo tudi konveksne krivulje.  Za začetek, pa moramo definirati množico premic v ravnini.

\begin{definicija}
Množico vseh premic na $\mathbb{R}^2$ definiramo kot: $$ \{ (p,\theta)|0 \le \theta \le 2\pi , p \ge 0\}$$, kjer $p$ predstavlja oddaljenost premice od središča, $\theta$ pa kot premice 
\end{definicija}



Definirajmo:
\begin{itemize}
\item parametre:
\begin{itemize}
\item $ d_i $: prvotna stopnja povpraševanja po izdelku $i$,
\item $ w_{ij}$: delež povpraševanja izdelka $j$ prenesenega na izdelek $i$, če $j$-tega izdelka ni v ponudbi; $w_{ii} = 1$,
\item $ v_i $: dobiček pri $i$-tem izdelku,
\item $ h_i $: strošek hranjenja izdelka na polici,
\item $ k_i $: cena polnjenja izdelka $i$, 
\item $ \theta_i $: koeficient količine varnostne zaloge izdelka $i$.
\end{itemize}

\item odločitvene spremenljivke:
\begin{itemize}
\item $ y_i $: indikator, ki nam pove, ali je izdelek $i$ vključen v ponudbo,
\item $ Q_i $: število naročenih izdelkov $i$.
\end{itemize}

\item pomožne spremenljivke:
\begin{itemize}
\item $ x_{ij} $: indikator substitucije iz izdelka $j$ na izdelek $i$; $x_{ij} = y_i (1-y_j)$, če $j \ne i$ in $x_{ij} = y_i$, če  $j = i$,
\item $ s_i $: končna efektivna stopnja povpraševanja po $i$-tem izdelku; $s_i = y_i (d_i + \sum_{j \ne i} w_{ij} d_j (1-y_j)) = \sum_j  w_{ij} d_j x_{ij}.$
\end{itemize} 
\end{itemize}

 Pri problemu bomo predpostavili neodvisno polnjenje polic, torej da  ima vsak izdelek svoj strošek polnjenja. Lahko bi uporabili tudi strategijo kombiniranega polnjenja polic, pri kateri dodamo še strošek polnjenja izdelka iz določene skupine. Bolj natančno, če dopolnimo zalogo vsaj enega izdelka iz skupine, plačamo še ta dodaten strošek.


Najosnovnejši problem je neomejen problem neodvisnega polnjenja polic. Naš cilj je maksimizacija dobička:

$$  \max_{y, Q,  s, x}  \sum_{i \in \mathcal{P}} ( v_i s_ i - h_i (\frac{Q_i }{ 2} + \theta_i Q_i) - \frac{k_i s_i}{Q_i})   $$ 
p.p.

 $ s_i = \sum_j w_{ij} d_j x_{ij}, \forall i$,

$ x_{ij} \leq y_i, \forall i, j  $,

$ x_{ij} \leq 1 - y_j, \forall i \ne j$,

$ y_i \in \{0,1\}$,

$ x_{ij} \geq 0, \forall i,j$,

$Q_i \geq 0, \forall i$.

\vspace*{4 mm}
Maksimiziramo torej neto dobiček (celoten dobiček zmanjšan za stroške hranjenja izdelka na polici in polnjenja izdelka), kjer prvi pogoj definira končno efektivno stopnjo povpraševanja, drugi in tretji pogoj definirata $x$ s pomočjo $y$, ostali pogoji pa določijo definicijsko območje $y$, $Q$ in $x$.

 
\section{Načrt dela}

V projektu se bova osredotočila na primerjavo opisanih strategij in kdaj je uporaba določene strategije primernejša. V nadaljnjem bova neomejen model nadgradila z omejenim, pri katerem uvedemo še dodatna parametra, ki določata ves skupni prostor v trgovini in porabo prostora posameznega izdelka. Zaradi računske težavnosti problema bova predstavila metode, ki olajšajo reševanje. Generirala bova naključne podatke in primerjala obe strategiji polnjenja polic v različnih okoliščinah. Delo bo potekalo v programskem jeziku Matlab.


\end{document}